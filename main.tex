%%%%%%%%%%%%%%%%%%%%%%%%%%%%%%%%%%%%%%%%%%%%%%%%%%%%%%%%%%%%%%%%%%%%%%
% LaTeX Template: Curriculum Vitae
%
% Source: http://www.howtotex.com/
% Feel free to distribute this template, but please keep the
% referal to HowToTeX.com.
% Date: July 2011
% 
%%%%%%%%%%%%%%%%%%%%%%%%%%%%%%%%%%%%%%%%%%%%%%%%%%%%%%%%%%%%%%%%%%%%%%
% How to use writeLaTeX: 
%
% You edit the source code here on the left, and the preview on the
% right shows you the result within a few seconds.
%
% Bookmark this page and share the URL with your co-authors. They can
% edit at the same time!
%
% You can upload figures, bibliographies, custom classes and
% styles using the files menu.
%
% If you're new to LaTeX, the wikibook is a great place to start:
% http://en.wikibooks.org/wiki/LaTeX
%
%%%%%%%%%%%%%%%%%%%%%%%%%%%%%%%%%%%%%%%%%%%%%%%%%%%%%%%%%%%%%%%%%%%%%%
\documentclass[paper=a4,fontsize=11pt,norsk]{scrartcl} % KOMA-article class
							

\usepackage[english]{babel}
\usepackage[utf8x]{inputenc}
\usepackage[protrusion=true,expansion=true]{microtype}
\usepackage{amsmath,amsfonts,amsthm}     % Math packages
\usepackage{graphicx}                    % Enable pdflatex
\usepackage[svgnames]{xcolor}            % Colors by their 'svgnames'
\usepackage{geometry}
	\textheight=700px                    % Saving trees ;-)
\usepackage{url}
\usepackage{hyperref}
\hypersetup{
    colorlinks=true,
    linkcolor=blue,
    filecolor=magenta,      
    urlcolor=cyan,
}
\urlstyle{same}

\frenchspacing              % Better looking spacings after periods
\pagestyle{empty}           % No pagenumbers/headers/footers

%%% Custom sectioning (sectsty package)
%%% ------------------------------------------------------------
\usepackage{sectsty}

\sectionfont{%			            % Change font of \section command
	\usefont{OT1}{phv}{b}{n}%		% bch-b-n: CharterBT-Bold font
	\sectionrule{0pt}{0pt}{-5pt}{3pt}}

%%% Macros
%%% ------------------------------------------------------------
\newlength{\spacebox}
\settowidth{\spacebox}{8888888888}			% Box to align text
\newcommand{\sepspace}{\vspace*{1em}}		% Vertical space macro

\newcommand{\MyName}[1]{ % Name
		\Huge \usefont{OT1}{phv}{b}{n} \hfill #1
		\par \normalsize \normalfont}
		
\newcommand{\MySlogan}[1]{ % Slogan (optional)
		\large \usefont{OT1}{phv}{m}{n}\hfill \textit{#1}
		\par \normalsize \normalfont}

\newcommand{\NewPart}[1]{\section*{\uppercase{#1}}}

\newcommand{\PersonalEntry}[2]{
		\noindent\hangindent=2em\hangafter=0 % Indentation
		\parbox{\spacebox}{        % Box to align text
		\textit{#1}}		       % Entry name (birth, address, etc.)
		\hspace{1.5em} #2 \par}    % Entry value

\newcommand{\SkillsEntry}[2]{      % Same as \PersonalEntry
		\noindent\hangindent=2em\hangafter=0 % Indentation
		\parbox{\spacebox}{        % Box to align text
		\textit{#1}}			   % Entry name (birth, address, etc.)
		\hspace{1.5em} #2 \par}    % Entry value	
		
\newcommand{\EducationEntry}[4]{
		\noindent \textbf{#1} \hfill      % Study
		\colorbox{Black}{%
			\parbox{6em}{%
			\hfill\color{White}#2}} \par  % Duration
		\noindent \textit{#3} \par        % School
		\noindent\hangindent=2em\hangafter=0 \small #4 % Description
		\normalsize \par}

\newcommand{\WorkEntry}[4]{				  % Same as \EducationEntry
		\noindent \textbf{#1} \hfill      % Jobname
		\colorbox{Black}{\color{White}#2} \par  % Duration
		\noindent \textit{#3} \par              % Company
		\noindent\hangindent=2em\hangafter=0 \small #4 % Description
		\normalsize \par}

%%% Begin Document
%%% ------------------------------------------------------------
\begin{document}
% You can upload a photo and include it here...
%\begin{wrapfigure}{l}{0.5\textwidth}
%	\vspace*{-2em}
%		\includegraphics[width=0.15\textwidth]{photo}
%\end{wrapfigure}

\MyName{Jørn Bøni Hofstad}
\MySlogan{Curriculum Vitae}

\sepspace

%%% Personal details
%%% ------------------------------------------------------------
\NewPart{Personal details}{}

\PersonalEntry{Birth}{June 7, 1996}
%\PersonalEntry{Address}{Hans Hagerups gate 6 7012 Trondheim}
\PersonalEntry{Phone}{+47 94121127}
\PersonalEntry{Mail}{\url{jorn.boni.hofstad@gmail.com}}

%%% Education
%%% ------------------------------------------------------------
\NewPart{Education}{}

\EducationEntry{M.Sc. Cybernetics and Robotics}{2015-2020}{Norwegian University of Science and Technology (NTNU)}{}
\EducationEntry{Exchange semester}{February - June 2019}{École Polytechnique Fédérale de Lausanne (EPFL)}{}

%%% Work experience
%%% ------------------------------------------------------------
\NewPart{Work experience}{}

\EducationEntry{Develop and maintain tooling for working with Nordic Semiconductor development boards}{Nordic Semiconductor}{2020-present}{

Implement cross-platform Continuous integration build systems used to make tools directly used by customers of Nordic Semiconductor. Implementation of a local testing system in Rust, which allowed for local testing consistent with tests performed on remote test servers. Add tool for listing and filtering development boards. Maintained and migrated two python projects to the newer versions of python 3.9 and 3.10.


}

% Terse version of SINTEF
\EducationEntry{Making a tool for comparing hydropower-production}{SINTEF Energy}{Summer 2019}{
Making a domain-specific tool for translating outputs from two hydropower-production optimization programs, Translating both concepts and formats in under-documented code.
}

% More detailed version of SINTEF
%\EducationEntry{Making a tool for comparing hydropower-production}{SINTEF Energy}{Summer 2019}{
%SINTEF Energy has two programs for planning how to most efficiently drain water reservoirs to get %the most out of the fluctuating water prices. Since one was a long-term model, while the other one was %more detailed, the job consisted of learning the models well enough, so that an automated program could be %made for translating the results from SHOP into the same format as the one for Prodrisk.  
%}

% %Short version of Nordic summer job
% \EducationEntry{Developing a simple BLE-mesh simulator}{Summer 2018}{Nordic Semiconductor Summer-job}{
% Developed simulator and presentation of results of a simple simulator used to estimate packet loss and throughput of a BLE mesh. 

% }
% More detailed version of the Nordic summer job
%\EducationEntry{Developing a simple BLE-mesh simulator}{Summer 2018}{Nordic Semiconductor Summer-job}{
%Developing a simple simulator with GUI for package transmission for Bluetooth Low energy mesh so that % customers could get an idea of the performance of a star topology of BLE nodes. The job %included setting up experiments with NRF51 boards to gather data on the number of received %packets, and some of their properties. As well as setting up a simulator in Python able to give %distributions and percentiles for discovery time and throughput, given a given set of parameters given by %the user (Jitter, number of nodes, packet-length, etc.) 
%}

%\EducationEntry{Student assistant (TTK4115, Linear System Theory )}{Fall 2018}{NTNU Part-time}
%{Student assistant in "Linear system theory". Primarily at the %helicopter labs, helping students with controlling a model helicopter by using a multivariable linear regulator %with state estimators. (I also corrected exercises) 
%}

%Sanntids-studass 
\EducationEntry{Student assistant (TTK4145, Real-time programming)}{Spring 2020}{NTNU Part-time}
{Assisting students in designing a robust distributed system, controlling a set of model elevators taking orders. The system was supposed to function even under limited handle hardware failures and handle network issues}

%Indel-studass 
\EducationEntry{Student assistant (TTK4240, Industrial Electronics)}{Fall 2017}{NTNU Part-time}
{Asisting students in modeling and measuring an electric motor,  with a discrete encoder, as well as balancing an inverted pendulum using a combination of an energy-based and an angle-based PID controller.
}

%%% Skills
%%% ------------------------------------------------------------
\NewPart{Skills}{}

\SkillsEntry{Languages}{Norwegian (native tongue)}
\SkillsEntry{}{English (C2)}
\SkillsEntry{}{Swiss German B2-C1 orally,  A1-A2 written) }
\SkillsEntry{}{French (A2-B1)}

\SkillsEntry{Software}{Rust, C++, Python, Elixir, \textsc{Matlab}}
\SkillsEntry{Other}{Driver's license}

%%% Hobbies
%%% ------------------------------------------------------------


%%% References
%%% ------------------------------------------------------------
\NewPart{References}{}
Can be provided on request
%\PersonalEntry{Name}{Hans Olav}
%\PersonalEntry{Company}{Energisystemer, SINTEF Energi}
%\PersonalEntry{Mail}{\url{hans.hagenvik@sintef.no}}
%\PersonalEntry{Mobile}{48004524}


%%% Links
\NewPart{Linkedin}{
\url{www.linkedin.com/in/joern-boeni-hofstad}
}

\NewPart{github}{
\url{https://github.com/jornbh/}
\end{document}
