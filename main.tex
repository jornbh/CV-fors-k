%%%%%%%%%%%%%%%%%%%%%%%%%%%%%%%%%%%%%%%%%%%%%%%%%%%%%%%%%%%%%%%%%%%%%%
% LaTeX Template: Curriculum Vitae
%
% Source: http://www.howtotex.com/
% Feel free to distribute this template, but please keep the
% referral to HowToTeX.com.
% Date: July 2011
%
%%%%%%%%%%%%%%%%%%%%%%%%%%%%%%%%%%%%%%%%%%%%%%%%%%%%%%%%%%%%%%%%%%%%%%
% How to use writeLaTeX:
%
% You edit the source code here on the left, and the preview on the
% right shows you the result within a few seconds.
%
% Bookmark this page and share the URL with your co-authors. They can
% edit at the same time!
%
% You can upload figures, bibliographies, custom classes and
% styles using the files menu.
%
% If you're new to LaTeX, the wikibook is a great place to start:
% http://en.wikibooks.org/wiki/LaTeX
%
%%%%%%%%%%%%%%%%%%%%%%%%%%%%%%%%%%%%%%%%%%%%%%%%%%%%%%%%%%%%%%%%%%%%%%
\documentclass[paper=a4,fontsize=11pt,norsk]{scrartcl} % KOMA-article class


\usepackage[english]{babel}
\usepackage[utf8x]{inputenc}
\usepackage[protrusion=true,expansion=true]{microtype}
\usepackage{amsmath,amsfonts,amsthm}     % Math packages
\usepackage{graphicx}                    % Enable pdflatex
\usepackage[svgnames]{xcolor}            % Colors by their 'svgnames'
\usepackage{geometry}
	\textheight=700px                    % Saving trees ;-)
\usepackage{url}
\usepackage{hyperref}
\hypersetup{
    colorlinks=true,
    linkcolor=blue,
    filecolor=magenta,
    urlcolor=cyan,
}
\urlstyle{same}

\frenchspacing              % Better looking spacings after periods
\pagestyle{empty}           % No pagenumbers/headers/footers

%%% Custom sectioning (sectsty package)
%%% ------------------------------------------------------------
\usepackage{sectsty}

\sectionfont{%			            % Change font of \section command
	\usefont{OT1}{phv}{b}{n}%		% bch-b-n: CharterBT-Bold font
	\sectionrule{0pt}{0pt}{-5pt}{3pt}}

%%% Macros
%%% ------------------------------------------------------------
\newlength{\spacebox}
\settowidth{\spacebox}{8888888888}			% Box to align text
\newcommand{\sepspace}{\vspace*{1em}}		% Vertical space macro

\newcommand{\MyName}[1]{ % Name
		\Huge \usefont{OT1}{phv}{b}{n} \hfill #1
		\par \normalsize \normalfont}

\newcommand{\MySlogan}[1]{ % Slogan (optional)
		\large \usefont{OT1}{phv}{m}{n}\hfill \textit{#1}
		\par \normalsize \normalfont}

\newcommand{\NewPart}[1]{\section*{\uppercase{#1}}}

\newcommand{\PersonalEntry}[2]{
		\noindent\hangindent=2em\hangafter=0 % Indentation
		\parbox{\spacebox}{        % Box to align text
		\textit{#1}}		       % Entry name (birth, address, etc.)
		\hspace{1.5em} #2 \par}    % Entry value

\newcommand{\SkillsEntry}[2]{      % Same as \PersonalEntry
		\noindent\hangindent=2em\hangafter=0 % Indentation
		\parbox{\spacebox}{        % Box to align text
		\textit{#1}}			   % Entry name (birth, address, etc.)
		\hspace{1.5em} #2 \par}    % Entry value

\newcommand{\EducationEntry}[4]{
		\noindent \textbf{#1} \hfill      % Study
		\colorbox{Black}{%
			\parbox{6em}{%
			\hfill\color{White}#2}} \par  % Duration
		\noindent \textit{#3} \par        % School
		\noindent\hangindent=2em\hangafter=0 \small #4 % Description
		\normalsize \par}

\newcommand{\WorkEntry}[4]{				  % Same as \EducationEntry
		\noindent \textbf{#1} \hfill      % Jobname
		\colorbox{Black}{\color{White}#2} \par  % Duration
		\noindent \textit{#3} \par              % Company
		\noindent\hangindent=2em\hangafter=0 \small #4 % Description
		\normalsize \par}

%%% Begin Document
%%% ------------------------------------------------------------
\begin{document}
% You can upload a photo and include it here...
%\begin{wrapfigure}{l}{0.5\textwidth}
%	\vspace*{-2em}
%		\includegraphics[width=0.15\textwidth]{photo}
%\end{wrapfigure}

\MyName{Jørn Bøni Hofstad}
\MySlogan{Curriculum Vitae}

\sepspace

%%% Personal details
%%% ------------------------------------------------------------
\NewPart{Personal details}{}

\PersonalEntry{Birth}{June 7, 1996}
%\PersonalEntry{Address}{Hans Hagerups gate 6 7012 Trondheim}
\PersonalEntry{Phone}{+47 94121127}
\PersonalEntry{Mail}{\url{jorn.boni.hofstad@gmail.com}}
\PersonalEntry{Citizenship}{Norwegian, Swiss}
\PersonalEntry{LinkedIn}{\url{www.linkedin.com/in/joern-boeni-hofstad}}


%%% Skills
%%% ------------------------------------------------------------
\NewPart{Skills}{}

\SkillsEntry{Languages}{Norwegian (native), English (fluent)}
% \SkillsEntry{}{}
\SkillsEntry{}{Swiss German (fluent orally, A1-A2 written), French (A2)}
% \SkillsEntry{}{}

\SkillsEntry{Software}{Rust, C++, Python, Git, Docker, Jenkins}
% \SkillsEntry{Other}{Driver's license}

%%% Work experience
%%% ------------------------------------------------------------
\NewPart{Work experience}{}


\WorkEntry{Software tester}{2023-present}{Nordic Semiconductor}
{I helped to increase our customer's confidence in our products by testing our tools made to do Device Firmware Updates (DFUs) and resetting Nordic Semiconductor's devices.
I made CI/CD systems for building the projects from the remote Git repositories, as well as testing them on proper hardware.
The tests were done on Windows and Ubuntu, using Jenkins and Bash, a framework based on pytest and pytest-bdd. There was also a focus on making tests that could be used across several Nordic-devices, and across different device families.
I also made Docker images for building and testing our software. Our CI/CD systems were made using Jenkins and Bash.
I also made an internal Rust registry that could be used for internally releasing proprietary Rust packages

%I helped increase our and our customer's confidence in the tools we made for doing DFUs and resetting nRF devices using Nordic's Python testing framework.
%The tests were done both on FPGAs and directly on hardware. I also worked on the CI/CD system using Jenkins/Bash for building and testing. Additionally, I made the Docker images we that ran the nodes controlling to the FPGAs.

% Tested tools used for firmware updates controlling nRF devices.
% - Verified that the changes worked as expected according to the product specification.
% - Testing on FPGA and on hardware.
% - Testing using a Bash/Python framework.
% - Created a build and a test system in Jenkins.
% - Creating Docker image to be used by Jenkins

}

% Main points for the autostore position
% Work
% C++ .
% Rust
% Git, Linux, Bash
% Tester: Testing in hardware, pytest, Python,
% Resuable tools that were made to be generic. like nrfutil, and how we made tests to be generic for the different chips
% nRF, Pytest, DFU, device resetting, etc.
% strace and managing large log-files
% async and threading(?)
% Education
% C++ as a TA

% "Byggern": CAN, UART, I2C, ATMega. Embedded C. Embedded boards and breadboards

% Using Header-files to make generic and more portable code
% Worked with registers, and some of the Nordic'specific parts.

% Mechanics and electronics.

% Elektro intro, and industriell elektroteknikk
% TilpDat, with OpAmps and resistors for a PI regulator.
% Fysikk 1. rotation, momentum, forces, basic thermo physics.
% Is it worth to mention Advanced Algorithms?

% Modelling and simulation: Simulating friction, complex dynamic systems and constrained systems


\WorkEntry{Software developer}{2020-2023}{Nordic Semiconductor}
{I developed tooling for Windows, MacOS and Ubuntu, used internally and by customers, and written in Rust. I also helped develop for other tools in C++.

The tools involved:
\par


-  A tool for making an isolated toolchain for developing with the nRF SDK. This tool heavily used async to get aroughly a doubling in speed.
\par



- A portable Rust/Python tool, made in a multi-team project.
\par



- A dynamic completion system. We profiled and optimized the Rust code using Visual Studio.
% One of the tools was a bundler for making an isolated toolchain used for working with the nRF SDK. The tool greatly simplified the initial installation for our first-time users and increasing reproducibility for customer support.
% Additionally, I made our dynamic completion system more usable, taking execution time from 1 second to 0.3 seconds, profiling using ProcMon and Visual Studio.
% I also helped on updating one of our legacy C++ projects.
% Finally. I worked on a multi-team project to make a Rust/Python program that used a built-in Python installation, while also limiting the scope of the project.
% The tools were written in .
% Also helped maintain legacy C++ project.
% - Helped maintain nrfjprog (Multiprocess C++ project)
% - Made and maintained Azure/Jenkins pipelines.
% - Cross-team project, integrating Python code to be integrated with Rust
% - Profiling and optimizing build tools and autocompletion systems.
}


% \WorkEntry{Developed and maintained tooling for working with Nordic devices}{2020-present}{Nordic Semiconductor}{
% % Short
% % I developed cross-platform tooling used internally and by customers for developing our devices.
% % This included testing, profiling and parallelizing Rust, Python and Bash programs.

% I developed tooling for Windows, MacOS and Ubuntu, used internally and by customers for developing our devices. This includes tooling for collecting and bundling a portable toolchain for developing with Nordic's SDK, used by most users of Nordic's newer chips. The role also involved CI/CD systems in Azure Pipelines and Jenkins, as well as managed the releases of the aforementioned toolchains. The role also involved debugging and fixing multiprocess legacy applications in C++ and Python, as well as creating Docker images for building our tools.
% I also worked on a cross-team project, which involved making a portable Python program that integrated with our Rust tool suite and package manager. I also worked with profiling and optimizing our build tools and our autocompletion system.

% % low
% % This included creating and migrating Continuous Integration systems, as well as migrating them to Azure Pipelines and Jenkins
% % I was responsible for maintaining and deprecating our Python/C++ libraries and tools.



% % ##Things I have done in Nordic
% % - testing device-lib
% % - toolchain-bundler
% % 	- Building and releasing the toolchain bundles until another team was able to take the job in CI
% % 	- Parallelizing and simplifying for performance gains
% % - pc-ble-driver-py
% % - nrfutil
% % - azure pipelines
% % - nrfutil completion
% % - nrfutil device guessing
% % - Profiling nrfutil device and completion
% % 	- Linux and Windows
% % - Debug Linux and Microsoft processes
% % - NoVel
% % - Make docker images for building
% % - nrfutil npm
% % 	- Making Python project portable, and make it work with nrfutil
% % 	- Profiling size usage, file access and how to reduce it
% % 	- Contributed to the python project
% % - Testing util and probe lib
% % - setting up Jenkins


% }


% Terse version of SINTEF
% \WorkEntry{Software developer, Summer student}{Summer 2019}{SINTEF Energy}{
% I made a Python tool for translating outputs from a short-term and a long-term hydropower-production optimization program, making them comparable.
% }

% More detailed version of SINTEF
%\WorkEntry{Making a tool for comparing hydropower-production}{SINTEF Energy}{Summer 2019}{
%SINTEF Energy has two programs for planning how to most efficiently drain water reservoirs to get %the most out of the fluctuating water prices. Since one was a long-term model, while the other one was %more detailed, the job consisted of learning the models well enough, so that an automated program could be %made for translating the results from SHOP into the same format as the one for Prodrisk.
%}

% %Short version of Nordic summer job
% \WorkEntry{Developing a simple BLE-mesh simulator}{Summer 2018}{Nordic Semiconductor Summer-job}{
% Developed simulator and presentation of results of a simple simulator used to estimate packet loss and throughput of a BLE mesh.

% }
% More detailed version of the Nordic summer job
%\WorkEntry{Developing a simple BLE-mesh simulator}{Summer 2018}{Nordic Semiconductor Summer-job}{
%Developing a simple simulator with GUI for package transmission for Bluetooth Low energy mesh so that % customers could get an idea of the performance of a star topology of BLE nodes. The job %included setting up experiments with NRF51 boards to gather data on the number of received %packets, and some of their properties. As well as setting up a simulator in Python able to give %distributions and percentiles for discovery time and throughput, given a given set of parameters given by %the user (Jitter, number of nodes, packet-length, etc.)
%}



% \WorkEntry{Participated in Nordic Semiconductors's social committe}{2022-present}{Nordic Semiconductor}
% {I organized and hosted social events like salary beers and social board game nights}

%Sanntids-studass
\WorkEntry{Teaching assistant (TTK4145, Real-time programming)}{Spring 2020}{NTNU Part-time}
{I aided students in designing a robust distributed system of model elevators on an unreliable network, using unreliable hardware, TCP/UDP and Erlang/Elixir or Go}



%Indel-studass
\WorkEntry{Teaching assistant (TTK4240, Industrial Electronics)}{Fall 2017}{NTNU Part-time}
{I assisted students in controlling an inverted pendulum using a discrete encoder and an electric motor using a non-linear controller they made in LabView. The job also involved helping with exercises relating to electrical impedance, as well as converters like buck- and boost converters.
}

\WorkEntry{Teaching assistant (TDT4102 - Object-Oriented Programming)}{Spring 2017}{NTNU Part-time}
{I helped students learn the basics of C++, object-oriented programming, templates and pointers/smart-pointers.
}


\NewPart{Relevant univerisity Courses}{}

\EducationEntry{TTK4155 - Embedded and Industrial Computer Systems Design}{Fall 2017}{NTNU}
{Made a system made of multiple microcontrollers, using I2C, UART, CAN protocols. The processor was an ATMega. Most components were connected using a breadboard, other components were an ADC, variable resistances and external flash. The firmware as made in C.
}


\EducationEntry{TFY4115 - Physics}{Fall 2016}{NTNU}
{Theoretical subject involving rotational and translational mechanics, momentum and thermodynamics.}






%%% Education
%%% ------------------------------------------------------------
\NewPart{Education}{}

\EducationEntry{M.Sc. Cybernetics and Robotics}{2015-2020}{Norwegian University of Science and Technology (NTNU), 3.8/5 GPA}{}
\EducationEntry{Exchange semester}{Spring 2019}{École Polytechnique Fédérale de Lausanne (EPFL)}{}



%%% Hobbies
%%% ------------------------------------------------------------


%%% References
%%% ------------------------------------------------------------
\NewPart{References}{}
Can be provided on request
%\PersonalEntry{Name}{Hans Olav}
%\PersonalEntry{Company}{Energisystemer, SINTEF Energi}
%\PersonalEntry{Mail}{\url{hans.hagenvik@sintef.no}}
%\PersonalEntry{Mobile}{48004524}


% %%% Links
% \NewPart{Linkedin}{
% \url{www.linkedin.com/in/joern-boeni-hofstad}
% }

\NewPart{GitHub}{}
\url{https://github.com/jornbh/}
\NewPart{Other}{}
Driver's License

\end{document}
