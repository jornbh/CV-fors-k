%%%%%%%%%%%%%%%%%%%%%%%%%%%%%%%%%%%%%%%%%%%%%%%%%%%%%%%%%%%%%%%%%%%%%%
% LaTeX Template: Curriculum Vitae
%
% Source: http://www.howtotex.com/
% Feel free to distribute this template, but please keep the
% referral to HowToTeX.com.
% Date: July 2011
%
%%%%%%%%%%%%%%%%%%%%%%%%%%%%%%%%%%%%%%%%%%%%%%%%%%%%%%%%%%%%%%%%%%%%%%
% How to use writeLaTeX:
%
% You edit the source code here on the left, and the preview on the
% right shows you the result within a few seconds.
%
% Bookmark this page and share the URL with your co-authors. They can
% edit at the same time!
%
% You can upload figures, bibliographies, custom classes and
% styles using the files menu.
%
% If you're new to LaTeX, the wikibook is a great place to start:
% http://en.wikibooks.org/wiki/LaTeX
%
%%%%%%%%%%%%%%%%%%%%%%%%%%%%%%%%%%%%%%%%%%%%%%%%%%%%%%%%%%%%%%%%%%%%%%
\documentclass[paper=a4,fontsize=11pt,norsk]{scrartcl} % KOMA-article class


\usepackage[english]{babel}
\usepackage[utf8x]{inputenc}
\usepackage[protrusion=true,expansion=true]{microtype}
\usepackage{amsmath,amsfonts,amsthm}     % Math packages
\usepackage{graphicx}                    % Enable pdflatex
\usepackage[svgnames]{xcolor}            % Colors by their 'svgnames'
\usepackage{geometry}
	\textheight=700px                    % Saving trees ;-)
\usepackage{url}
\usepackage{hyperref}
\hypersetup{
    colorlinks=true,
    linkcolor=blue,
    filecolor=magenta,
    urlcolor=cyan,
}
\urlstyle{same}

\frenchspacing              % Better looking spacings after periods
\pagestyle{empty}           % No pagenumbers/headers/footers

%%% Custom sectioning (sectsty package)
%%% ------------------------------------------------------------
\usepackage{sectsty}

\sectionfont{%			            % Change font of \section command
	\usefont{OT1}{phv}{b}{n}%		% bch-b-n: CharterBT-Bold font
	\sectionrule{0pt}{0pt}{-5pt}{3pt}}

%%% Macros
%%% ------------------------------------------------------------
\newlength{\spacebox}
\settowidth{\spacebox}{8888888888}			% Box to align text
\newcommand{\sepspace}{\vspace*{1em}}		% Vertical space macro

\newcommand{\MyName}[1]{ % Name
		\Huge \usefont{OT1}{phv}{b}{n} \hfill #1
		\par \normalsize \normalfont}

\newcommand{\MySlogan}[1]{ % Slogan (optional)
		\large \usefont{OT1}{phv}{m}{n}\hfill \textit{#1}
		\par \normalsize \normalfont}

\newcommand{\NewPart}[1]{\section*{\uppercase{#1}}}

\newcommand{\PersonalEntry}[2]{
		\noindent\hangindent=2em\hangafter=0 % Indentation
		\parbox{\spacebox}{        % Box to align text
		\textit{#1}}		       % Entry name (birth, address, etc.)
		\hspace{1.5em} #2 \par}    % Entry value

\newcommand{\SkillsEntry}[2]{      % Same as \PersonalEntry
		\noindent\hangindent=2em\hangafter=0 % Indentation
		\parbox{\spacebox}{        % Box to align text
		\textit{#1}}			   % Entry name (birth, address, etc.)
		\hspace{1.5em} #2 \par}    % Entry value

\newcommand{\EducationEntry}[4]{
		\noindent \textbf{#1} \hfill      % Study
		\colorbox{Black}{%
			\parbox{6em}{%
			\hfill\color{White}#2}} \par  % Duration
		\noindent \textit{#3} \par        % School
		\noindent\hangindent=2em\hangafter=0 \small #4 % Description
		\normalsize \par}

\newcommand{\WorkEntry}[4]{				  % Same as \EducationEntry
		\noindent \textbf{#1} \hfill      % Jobname
		\colorbox{Black}{\color{White}#2} \par  % Duration
		\noindent \textit{#3} \par              % Company
		\noindent\hangindent=2em\hangafter=0 \small #4 % Description
		\normalsize \par}

%%% Begin Document
%%% ------------------------------------------------------------
\begin{document}
% You can upload a photo and include it here...
%\begin{wrapfigure}{l}{0.5\textwidth}
%	\vspace*{-2em}
%		\includegraphics[width=0.15\textwidth]{photo}
%\end{wrapfigure}

\MyName{Jørn Bøni Hofstad}
\MySlogan{Curriculum Vitae}

\sepspace

%%% Personal details
%%% ------------------------------------------------------------
\NewPart{Personal details}{}

\PersonalEntry{Birth}{June 7, 1996}
%\PersonalEntry{Address}{Hans Hagerups gate 6 7012 Trondheim}
\PersonalEntry{Phone}{+47 94121127}
\PersonalEntry{Mail}{\url{jorn.boni.hofstad@gmail.com}}
\PersonalEntry{Citizenship}{Norwegian, Swiss}
\PersonalEntry{LinkedIn}{\url{www.linkedin.com/in/joern-boeni-hofstad}}


%%% Skills
%%% ------------------------------------------------------------
\NewPart{Skills}{}

\SkillsEntry{Languages}{Norwegian (native), English (fluent)}
% \SkillsEntry{}{}
\SkillsEntry{}{Swiss German (fluent orally, A1-A2 written), French (A2)}
% \SkillsEntry{}{}

\SkillsEntry{Programming}{
    % Good
    Rust+Cargo,
    Python,
    C/C++,
    Matlab/Simulink,
}
	\SkillsEntry{}{
   Jenkins,
   Azure pipelines,
	}
	\SkillsEntry{}{
		Bash,
		PowerShell,
		Erlang/Elixir,
		regex
}

\SkillsEntry{Tools}{
    pytest,
    strace,
    GNU coreutils,
    Git,
    Docker,
	}
\SkillsEntry{}{
   pytest-bdd,
   vim,
   Procmon,
   vcpkg,
   Jira
   }
\SkillsEntry{}{
   Git,
   Bitbucket,
   GitHub,
   Docker,
   }
\SkillsEntry{}{
	Windbg,
	RESTful APIs,
	html,
}

\SkillsEntry{Concepts}{
	Software testing,
	Scrum,
    Object-Oriented Programming
}
\SkillsEntry{}{
	Embedded development,
	Linux system calls, signals and processes,
	}
\SkillsEntry{}{
	Agile development
	Concurrency and async code,
	%    Dependency Injection,
	}
\SkillsEntry{}{
	Functional programming

%    % Medium, and I kind of understand it
%    OAuth,
%    JSON/YAML/TOML,
%    Intel hex-format, % The most important parts, and how it is structured
%    %% Kind of bad
%    tkinter,
}


%%% Work experience
%%% ------------------------------------------------------------
\NewPart{Work experience}{}


%%%%%%%%%%%%%%%%%%%%%%%%%%%%%%%%%%%%%%%%%%%%%%%%%%%%%%%%%%%%%%%%%%%%%%%%%%%%%%%%%%%%%%%%%%%%%%%%%%%%%%%%%%%%%%%


\WorkEntry{Test tools for flashing, resettign and erasing development kits}{2024}{Nordic Semiconductor} { I made tests for our libraries and our CLI-tools that integrated with our Development Kits. The test-frameworks were pytest/pytest-bdd and Nordic's internal test framework. Pydebug remote was used for simplifying development, even on remote nodes, even allowing debugging on CI if need be. Tests were run on proper hardware and on FPGAs, flashing the netslist using proFPGA. The job also involved being up-to-date on the different devices, some of their different behavior and making sure the Jenkins nodes were running properly }
\WorkEntry{Set up Jenkins for building and testing}{2024}{Nordic Semiconductor} { I set up our Jenkins test and build pipelines. Most of the logic was made using Bash to keep the logic more portable between different CI vendors. }
\WorkEntry{Made proof of concepts for OAuth login}{2023}{Nordic Semiconductor} { I made an OAuth proof of concept for signing in to Azure with a Microsoft using Rust. }
\WorkEntry{Made an internal Rust registry}{2023}{Nordic Semiconductor} { I made an internal Cargo(Rust) package registry for publishing packages to our internal Artifactory. By using the Semantic versioning in Cargo, I made an ordering that allowed for nightly releases. I also made a system using GNU coreutils and Cargo, so that Jenkins could update the Git-repository on GitHub. }
\WorkEntry{Made Python/Rust tool made for estimating power-usage}{2022}{Nordic Semiconductor} { Expert code made using Python Anaconda needed to be embeddable in our modular CLI-tool. I helped push for using PyInstaller for exposing the Python code, finishing the project much faster. I also profiled and optimized the file-size, by monitoring all accessed files using ProcMon and then replacing the large SciPy linear algebra libraries with the ones form pip. This took the file-size from 250MB to 70MB, making the program much more usable. }
\WorkEntry{Add parent-process monitoring to long-running CLI tools}{2022}{Nordic Semiconductor} { In order to keep one of our C++ cli-tools from having orphaned child-processes when testing, it was required that the child monitored the parent. To observe and document the issue, Strace and kill and "ps" were used to observe and trigger the behavior. Solutions were made for Unix using process groups. On Windows, job objects were used to kill the children if the parent died. The task also involved pushing back against this solution, as simpler solutions could be made using queues in Boost. }
\WorkEntry{Add device-guessing for our command-line tools}{2023}{Nordic Semiconductor} { Added support in our Rust CLI tool for automatically guess what device to perform an operation on if there is only one connected device that the operation can be performed on. The work involved JSON Serialization/De-serialization using Rust Serde. }
\WorkEntry{Complete and optimize automatic completion system}{2022}{Nordic Semiconductor} { I helped finishing up our command completion system, as well as fixing bugs. I also fixed poor performance on Windows, by profiling time-usage and the timing of system calls using "Process Monitor". Visual Studio was also used for profiling. }
\WorkEntry{Create a portable toolchains for the nRF Connect SDK}{2022}{Nordic Semiconductor} { Made a Rust build-tool for making a portable toolchain for developing for Nordic semiconductor's nRF Connect SDK (Software Development Kit). In order to make toolchains for the different operating systems, things like Homebrew and apt were used. Async Rust with tokio was used in some cases for a speed-up. Also, chroot and FakeRoot were used to make the toolchain on Linux. }
\WorkEntry{Migrate and maintain legacy Python projects}{2021}{Nordic Semiconductor} {Migrated hybrid Python/C++ from Python 3.8 to 3.9. I also fixed bugs related to the Python Global Interpreter Lock using the Visual Studio as a post-mortem debugger. The task also involved updating the Windows registry to set Visual Studio or WinDbg as the post-mortem debugger. The task also involved setting up the PyInstaller script, as well as making sure that the resulting portable executable functioned. The task also approving external pull-requests and making sure the project was in a good spot before it was retired. }
\WorkEntry{Migrated build-systems from Jenkins to Azure pipelines}{2020}{Nordic Semiconductor} {I migrated and created build-pipelines for some of our tools to Azure pipelines, performing installations of the necessary tools, like vcpkg on Microsoft build-nodes. The nodes were Linux, MacOS and Windows. The build-scripts were in the Azure pipelines format, and in Bash and PowerShell. }
\WorkEntry{Helped maintain and debug C++ library}{2020}{Nordic Semiconductor} { I made improvements and debugged Nordic Semiconductor's internal library for communicating with nRF development kits. The tool was built using CMake, vcpkg and ninja. In order to simplify getting an overview of the project Doxygen was also used. }
\WorkEntry{Summer intern embedded developer}{Summer 2018}{Nordic Semiconductor} {I made firmware to characterize the characteristics of receiving Bluetooth packages and characterizing package loss, written in C. The firmware was interrupt based and worked with the Direct Memory Access peripherals like the Bluetooth radio. It was also branchless to ensure consistent behavior. The results were made for making a simulator form estimating package loss, written in Python. The graphic surface made using tkinter. }
\WorkEntry{Summer researcher}{Summer 2019}{SINTEF Energy Research} { I made a Python tool for translating the output in hdf format form SINTEF's short-term tool for optimizing hydro-power production to be compatible with their long-term tool for optimizing the same. The job involved learning the differing concepts of hydro-power production, so that they could be translated properly. }



%%%%%%%%%%%%%%%%%%%%%%%%%%%%%%%%%%%%%%%%%%%%%%%%%%%%%%%%%%%%%%%%%%%%%%%%%%%%%%%%%%%%%%%%%%%%%%%%%%%%%%%%%%%%%%%




% TODO: Fill in the long CV of experiences and stuff:
% - SINTEF: Python, hdf5, SHOP, Prodrisk,
% - Nordic: numpy, tkinter, Bluetooth, nrf52, (exponential distribution)

%\WorkEntry{Software tester}{2023-present}{Nordic Semiconductor}{
%I helped increase our and our customer's confidence in the tools we made for doing DFUs and resetting nRF devices using Nordic's Python testing framework.
%The tests were done both on FPGAs and directly on hardware. I also worked on the CI/CD system using Jenkins/Bash for building and testing. Additionally, I made the Docker images we that ran the nodes controlling to the FPGAs.

% Tested tools used for firmware updates controlling nRF devices.
% - Verified that the changes worked as expected according to the product specification.
% - Testing on FPGA and on hardware.
% - Testing using a Bash/Python framework.
% - Created a build and a test system in Jenkins.
% - Creating Docker image to be used by Jenkins
% }

%\WorkEntry{Software developer}{2020-2023}{Nordic Semiconductor}{
%I developed tooling for Windows, MacOS and Ubuntu, used internally and by customers, written in Rust.
%One of the tools was a bundler for making an isolated toolchain used for working with the nRF SDK. The tool greatly simplified the initial installation for our first-time users and increasing reproducibility for customer support.
%Additionally, I made our dynamic completion system more usable, taking execution time from 1 second to 0.3 seconds, profiling using ProcMon and Visual Studio.
%I also helped on updating one of our legacy C++ projects.
%% Finally. I worked on a multi-team project to make a Rust/Python program that used a built-in Python installation, while also limiting the scope of the project.
%% The tools were written in .
%% Also helped maintain legacy C++ project.
%% - Helped maintain nrfjprog (Multiprocess C++ project)
%% - Made and maintained Azure/Jenkins pipelines.
%% - Cross-team project, integrating Python code to be integrated with Rust
%% - Profiling and optimizing build tools and autocompletion systems.
%}


% \WorkEntry{Developed and maintained tooling for working with Nordic devices}{2020-present}{Nordic Semiconductor}{
% % Short
% % I developed cross-platform tooling used internally and by customers for developing our devices.
% % This included testing, profiling and parallelizing Rust, Python and Bash programs.

% I developed tooling for Windows, MacOS and Ubuntu, used internally and by customers for developing our devices. This includes tooling for collecting and bundling a portable toolchain for developing with Nordic's SDK, used by most users of Nordic's newer chips. The role also involved CI/CD systems in Azure Pipelines and Jenkins, as well as managed the releases of the aforementioned toolchains. The role also involved debugging and fixing multiprocess legacy applications in C++ and Python, as well as creating Docker images for building our tools.
% I also worked on a cross-team project, which involved making a portable Python program that integrated with our Rust tool suite and package manager. I also worked with profiling and optimizing our build tools and our autocompletion system.

% % low
% % This included creating and migrating Continuous Integration systems, as well as migrating them to Azure Pipelines and Jenkins
% % I was responsible for maintaining and deprecating our Python/C++ libraries and tools.



% % ##Things I have done in Nordic
% % - testing device-lib
% % - toolchain-bundler
% % 	- Building and releasing the toolchain bundles until another team was able to take the job in CI
% % 	- Parallelizing and simplifying for performance gains
% % - pc-ble-driver-py
% % - nrfutil
% % - azure pipelines
% % - nrfutil completion
% % - nrfutil device guessing
% % - Profiling nrfutil device and completion
% % 	- Linux and Windows
% % - Debug Linux and Microsoft processes
% % - NoVel
% % - Make docker images for building
% % - nrfutil npm
% % 	- Making Python project portable, and make it work with nrfutil
% % 	- Profiling size usage, file access and how to reduce it
% % 	- Contributed to the python project
% % - Testing util and probe lib
% % - setting up Jenkins


% }


% Terse version of SINTEF
% \WorkEntry{Software developer, Summer student}{Summer 2019}{SINTEF Energy}{
% I made a Python tool for translating outputs from a short-term and a long-term hydropower-production optimization program, making them comparable.
% }

% %Short version of Nordic summer job
% \WorkEntry{Developing a simple BLE-mesh simulator}{Summer 2018}{Nordic Semiconductor Summer-job}{
% Developed simulator and presentation of results of a simple simulator used to estimate packet loss and throughput of a BLE mesh.

% }
% More detailed version of the Nordic summer job
%\WorkEntry{Developing a simple BLE-mesh simulator}{Summer 2018}{Nordic Semiconductor Summer-job}{
%Developing a simple simulator with GUI for package transmission for Bluetooth Low energy mesh so that % customers could get an idea of the performance of a star topology of BLE nodes. The job %included setting up experiments with NRF51 boards to gather data on the number of received %packets, and some of their properties. As well as setting up a simulator in Python able to give %distributions and percentiles for discovery time and throughput, given a given set of parameters given by %the user (Jitter, number of nodes, packet-length, etc.)
%}



\WorkEntry{Participated in Nordic Semiconductor's social committee}{2022-2024}{Nordic Semiconductor}
{I organized and hosted social events like salary beers and social board game nights}

\WorkEntry{Teaching assistant (TTK4145, Real-time programming)}{Spring 2020}{NTNU Part-time}
{I aided students in designing a robust distributed system of model elevators on an unreliable network, using unreliable hardware, TCP/UDP and Erlang/Elixir or Go}

\WorkEntry{Teaching assistant (TTK4215 - Adaptive Control)}{Fall 2019}{NTNU Part-time}
{I assisted students with theoretical exercises relating to dynamic parameter estimation for linear systems, Model Reference Adaptive Control (MRAC), adaptive backstepping and reinforcement learning. Some exercises were also simulations of the estimators and controllers in Matlab/Simulink. I also approved exercises.
}


\WorkEntry{Teaching assistant (TTK4115 -  Linear System Theory)}{Fall 2018}{NTNU Part-time}
{I assisted students in lab-work relating to controlling a model-helicopter using Linear Quadratic Regulators (LQR), which and without Integation-parts. State estimations were also done using Kalman-filters
}

\WorkEntry{Teaching assistant (TTK4240, Industrial Electronics)}{Fall 2017}{NTNU Part-time}
{I assisted students in controlling an inverted pendulum using a discrete encoder and an electric motor using a non-linear controller they made in LabView. The job also involved helping with exercises relating to electrical impedance, as well as converters like buck- and boost converters.
}

\WorkEntry{Teaching assistant (TDT4102 - Object-Oriented Programming)}{Spring 2017}{NTNU Part-time}
{I helped students learn the basics of C++, object-oriented programming, templates and pointers/smart-pointers.
}


\NewPart{Relevant univerisity Courses}{}


\EducationEntry{TTK4155 - Embedded and Industrial Computer Systems Design}{Fall 2017}{NTNU}
{Made a system made of multiple microcontrollers, using I2C, UART, CAN protocols. The processor was an ATMega. Most components were connected using a breadboard, other components were an ADC, variable resistances and external flash.
}


\EducationEntry{TDT4200 - Parallel Computing}{Fall 2018}{NTNU}
{
Measurement and profiling for optimization. Rasterisation. Programming Nvidia GPUs using CUDA, as well as profiling the code.
}

\EducationEntry{TDT4120 - Algorithms and Data Structures}{Spring 2018}{NTNU}
{The course involved topics like heaps, trees, shortest-path algorithms, minimum spanning trees, maximum flow algorithms and minimum cut algorithms. Breadth/Depth First Search (BFS/DFS), Sorting algorithms, Big O notation.
}



\EducationEntry{Information security and privacy}{Spring 2019}{EPFL}
{The course involved topics like SQL-injections, asymmetric encryption, side-channel attacks and buffer-overflow attacks. It also included homomorphic encryption, de-anonymization, zero-knowledge proofs. SQL-injections, cross-site scripting, side-channel attacks and buffer-overflow attacks.}


\EducationEntry{TFY4115 - Physics}{Fall 2016}{NTNU}
{Theoretical subject involving rotational and translational mechanics, momentum and thermodynamics.}

\EducationEntry{Signal processing for communication}{Spring 2019}{EPFL}
{Discrete Fourier Transform (DFT). Finite and infinite Impulse Response filters. Signal encoding using a carrier wave. High and low-pass filters}

\EducationEntry{Artificial Neural Networks}{Spring 2019}{EPFL}
{Classification. Artificial neural networks for reinforcement learning (Q-learning). Long-Short-Term Memory. Some exercises were done in Jupyter notebooks}

\EducationEntry{TDT4125 - Algorithm Construction}{Spring 2018}{NTNU}
{The course was for advanced algorithms beyond the normal algorithms and data structures course. The course involved solutions and approximations for NP-complete problems, like kernelization, Integer linear programming and Traveling Sales-Person in metric spaces. It also included matroids, streaming algorithms and adversarial problems, like k-server.}


\EducationEntry{Computer vision}{Spring 2019}{EPFL}
{The course involved topics edge detection, image segmentation and clustering}



%%% Education
%%% ------------------------------------------------------------
\NewPart{Education}{}

\EducationEntry{M.Sc. Cybernetics and Robotics}{2015-2020}{Norwegian University of Science and Technology (NTNU), 3.8/5 GPA}{}
\EducationEntry{Exchange semester}{Spring 2019}{École Polytechnique Fédérale de Lausanne (EPFL)}{}



%%% Hobbies
%%% ------------------------------------------------------------


%%% References
%%% ------------------------------------------------------------
\NewPart{References}{}
Can be provided on request
%\PersonalEntry{Name}{Hans Olav}
%\PersonalEntry{Company}{Energisystemer, SINTEF Energi}
%\PersonalEntry{Mail}{\url{hans.hagenvik@sintef.no}}
%\PersonalEntry{Mobile}{48004524}


% %%% Links
% \NewPart{Linkedin}{
% \url{www.linkedin.com/in/joern-boeni-hofstad}
% }

\NewPart{GitHub}{}
\url{https://github.com/jornbh/}
\NewPart{Other}{}

\begin{itemize}
% [topsep=8pt,itemsep=4pt,partopsep=4pt, parsep=4pt]
\item Driver's License
\item Formerly Scrum certified

\end{itemize}

\end{document}
